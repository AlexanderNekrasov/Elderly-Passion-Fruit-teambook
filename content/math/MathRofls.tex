$ClassesCount = \dfrac{1}{|G|} \sum_{\pi \in G} I(\pi)$\\
$ClassesCount = \dfrac{1}{|G|} \sum_{\pi \in G} k^{C(\pi)}$\\
$\binom{n}{k} \equiv \prod_{i}\binom{n_i}{k_i}$, $n_i, k_i$ %- цифры $n,k$ в p-ичной системе счисления
\\
$\int_{a}^{b}f(x)dx \approx \dfrac{b-a}{6}(f(a) + 4f(\dfrac{a+b}
{2})+f(b))$\\
\\
$x_{n+1} = x_n - \dfrac{f(x_n)}{f'(x_n)}$, $O(loglog)$\\
\\
$G(n) = n \oplus (n >> 1)$\\
\begin{lstlisting}
int rev_g (int g) {
	int n = 0;
	for (; g; g >>= 1)
		n ^= g;
	return n;
}
\end{lstlisting}\\
\\
$g(n) = \sum_{d|n} f(d) \Rightarrow f(n) = \sum_{d|n} g(d) \mu(\frac{n}{d})$\\\\
$\sum_{d|n} \mu(d) = [n = 1],   \mu(1) = 1, \mu(p) = -1, \mu(p^k) = 0$\\
$\sin(a \pm b) = \sin a \cos b \pm \sin b \cos a$\\
$\cos(a \pm b) = \cos a \cos b \mp \sin a \sin b$\\
$\tg(a\pm b) = \dfrac{\tg a \pm \tg b}{1 \mp \tg a \tg b}$\\
$\ctg(a \pm b) = \dfrac{\ctg a \ctg b \mp 1}{\ctg b \pm \ctg a}$\\
$\sin \dfrac{a}{2} = \pm \sqrt{\dfrac{1-\cos a}{2}}$\\
$\cos \dfrac{a}{2} = \pm \sqrt{\dfrac{1+\cos a}{2}}$\\
$\tg \dfrac{a}{2} = \dfrac{\sin a}{1 - \cos a} = \dfrac{1 - cos a}{sin a}$\\
$\sin a \sin b = \dfrac{\cos(a-b) - \cos(a+b)}{2}$\\
$\sin a \cos b = \dfrac{\sin(a-b) + \sin(a+b)}{2}$\\
$\cos a \cos b = \dfrac{\cos(a-b) + \cos(a+b)}{2}$\\
%1 января 2000 года - суббота, 1 января 1900 года - понедельник, 14 апреля 1961 года - пятница
